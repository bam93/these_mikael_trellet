\begin{titlepage}

\includegraphics[height=2.cm]{./figures/logo_paris_saclay_stic}\hfill
\includegraphics[height=2.cm]{./figures/limsilogo_new_transparent_crop}\hfill
\\
\\



\begin{center}
 \textbf{THÈSE DE DOCTORAT DE\\ L'UNIVERSITÉ PARIS SUD\\}
\vspace{\stretch{0.5}}
Spécialité\,:\\
\textbf{Informatique}\\ 
\vspace{\stretch{0.5}}
Présentée par\,:\\ 
\vspace{\stretch{0.5}}
\begin{LARGE}
Mikael Trellet\end{LARGE}\\
\vspace{\stretch{1}}
Pour obtenir le grade de\\
\textbf{DOCTEUR DE L'UNIVERSITÉ PARIS SUD}
\end{center}

\vspace{\stretch{2}}
\noindent \underline{Sujet de la thèse\,:}\\
\begin{center}
\begin{Large}
{\textsc{Exploration et analyse immersive de données moléculaires guidés par la tâche et leur modélisation sémantique}}
\end{Large}
\end{center}

\vspace{\stretch{2}}
Soutenue le vendredi 18 décembre 2015\\

devant le jury composé de :\\
\begin{center}
	\begin{tabular}{l l l}
	
	Indira Thouvenin 	& Enseignant-chercheur HDR / UMR 7253 Heudiasyc		& Rapporteur\\ 
	Serge Pérez		& Directeur de recherche émérite CNRS / UMR 5063 DPM		& Rapporteur\\ %
	& &\\
	Alexandre Bonvin	& Professeur Utrecht university / Bijvoet Center	 	& Examinateur\\ 
	Nicolas Sabouret	& Professeur CNRS / LIMSI				& Examinateur\\ 
	& &\\	
	Patrick Bourdot 	& Directeur de Recherche CNRS / LIMSI				& Directeur de thèse\\ 
	Marc Baaden 	& Directeur de Recherche CNRS / LBT				& Directeur de thèse\\ 
	
	\end{tabular}
\end{center}

\vspace{\stretch{2}}


% \clearpage
% \newpage
% \thispagestyle{empty}   

% \mbox{~} % ou tout simplement l'espace insécable ~ mais on risque de l'oublier.

% \vfill 

\setlength{\columnsep}{7mm}
\setlength{\columnseprule}{0pt}

\begin{multicols}{2} 
\small 
\noindent Groupe VENISE	\\	
\noindent LIMSI-CNRS					\\
\noindent B.P. 133				\\
\noindent 91403 Orsay Cedex, France \\	

\columnbreak

\raggedleft LBT \\
\noindent IBPC-CNRS \\
\noindent 13 Rue Pierre et Marie Curie  \\
\noindent 75005 Paris, France
\end{multicols}



\end{titlepage}