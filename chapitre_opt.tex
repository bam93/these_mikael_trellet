\chapter[Observation de structures 3D sur dispositifs mobiles]{Simuler et contrôler grâce à des écrans déportés pour préserver l'immersion}
\label{Sec:visuAna}
\minitoc
\cleardoublepage

%% Commentaire : la commande \texorpdfstring permet de déclarer un titre de
%% chapitre (ou section, sous-section) alternatif en texte seul, si besoin, qui
%% est utilisé par hyperref pour fabriquer un menu dans les fichiers compilés

%\chapter{\texorpdfstring{Contrôle gestuel de l'articulation}{Contrôle gestuel de l'articulation}}
%% Commentaire : la commande \texorpdfstring permet de déclarer un titre de
%% chapitre (ou section, sous-section) alternatif en texte seul, si besoin, qui
%% est utilisé par hyperref pour fabriquer un menu dans les fichiers compilés

%Exemple de notation qui sera reprise dans l'index : soit $\Q$\index{Q@$\Q$} le corps des nombres rationnels.

\section{Introduction}

\section{Créer un espace de visualisation temps réel sur simulation live}

\section{Contrôler l'avancement de la simulation via snapshots 3D / Partager et diffuser information}