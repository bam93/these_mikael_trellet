%\chapterstar{INTRODUCTION}
% Pour des renseignements sur \chapterstar : voir le fichier macros.tex

\chapter*{Introduction} % "*" pour que l'introduction ne s'affiche pas dans la table des matières, sinon elle s'y affichera comme un chapitre d
\mtcaddchapter
\addstarredpart{INTRODUCTION} % Pour ajouter une partie ("part") fictive dans la table des matières
\mtcaddpart
\markboth{Introduction}{Introduction}  %% header manuel car sinon, ils me marquent le header du dernier \chapter{?} (on est dans un \chapter*{?} )
\selectlanguage{francais}

L'introduction nous permettra tout d'abord de poser le contexte et la problématique liés au sujet de la thèse. Dans un deuxième temps, nous exposerons brièvement l'approche générale choisie afin de répondre à cette problématique et enfin nous présenterons un plan général du manuscrit.

\subsection*{Contexte et problématique}

En science, la performance croissante des outils de calcul informatique permet aujourd'hui de générer des volumes de données considérables. Ceci est particulièrement vrai pour le domaine de la biologie structurale qui voit ses modèles moléculaires croître rapidement, à la fois en précision et en taille. Cette complexité grandissante est rendue possible par l'efficacité croissante des programmes de simulation. Ces derniers peuvent désormais générer des images décrivant l'évolution temporelle à une précision atomique pour des structures moléculaires pouvant atteindre plusieurs millions de particules. L'augmentation des moyens de calcul n'est malheureusement pas directement corrélé à l'augmentation des moyens d'analyse et la quantité générée de données est bien souvent très supérieure à la quantité de données traitée par les experts scientifiques. De même, les capacités de calcul ont depuis longtemps dépassé les capacités de stockage, pourtant elles-mêmes en croissance permanente. Enfin, la taille des données impacte directement l'efficacité et la rapidité des communications entre les machines impliquées dans le processus de calcul. La minimisation des échanges de données, de la quantité de données à stocker et à analyser suite à d'importants calculs est donc aujourd'hui un enjeu crucial pour le traitement des données scientifiques.

Il est possible de diviser la biologie structurale et donc l'étude théorique de structures moléculaires en quatre activités principales organisées selon le processus séquentiel suivant: la collecte de données expérimentales ou théoriques, la visualisation des structures 3d, la simulation moléculaire, l’analyse et l’interprétation des résultats. Il n'est pas rare que plusieurs itérations de ce processus soient nécessaires à l'établissement d'une conclusion scientifique significative pour le phénomène biologique étudié. Chaque nouveau processus prenant comme données d'entrée une partie des résultats générés par l'analyse des données du processus précédent.  

L'ajout d'un moyen de contrôle sur la génération des données, parallèlement à la simulation, pourrait permettre de filtrer les données obtenues. Ce contrôle doit impérativement être sans pertes pour la compréhension finale du phénomène étudié. L'approche \textit{in situ} essaye de mettre en place des outils pour le calcul, l'analyse et le rendu de données de simulation numérique en mettant en avant des solutions technologiques de haute performance permettant le suivi en temps interactif d'une simulation moléculaire. Cette synchronisation entre une simulation tournant en arrière-plan, son rendu visuel et les analyses pouvant se greffer dessus permet à l'utilisateur de diriger sa simulation suivant les données d'analyse qu'il possède déjà.
Il est donc possible dans une certaine mesure d'automatiser ce filtrage des données mais il reste un besoin évident pour l'expertise humaine afin d'avoir un impact réel et positif. 
\commentaire{Faire attention ici on sous entend que des approches pour résoudre ce problème n'existent pas, alors que certaine de type calcul/analyse/rendu InSitu essayent de résoudre le problème, c'est la différence entre notre approche et InSitu qu'il faut appuyer}
\correction{Ok}

La volonté de mettre l'expert scientifique au centre de la boucle de décision passe par la mise en place d'éléments d'analyse et de visualisation appropriés. Ceux-ci doivent permettre de présenter les données de façon à ce que l'utilisateur puisse appréhender dans les meilleures conditions le phénomène observé. Ces deux composantes, visuelles et analytiques, amènent avec elles des problématiques précises auxquelles il est nécessaire de répondre pour pouvoir implémenter tout élément supplémentaire de contrôle. Tout d'abord, la taille des systèmes moléculaires ne cessant d'augmenter, la visualisation de ceux-ci doit passer par des espaces plus importants et plus adaptés.

Le traitement analytique des données se fait en parallèle de leur observation et peut constituer un espace indépendant mais se doit cependant d'être cohérent avec le rendu visuel des données. La dimension immersive apporte une première réponse à ces problèmes en fournissant un espace d'affichage suffisamment important pour traiter des complexes moléculaires entiers dans leur environnement. Cet espace d'affichage plus important se fait soit par des dispositifs de grande taille où l'utilisateur peut évoluer tout en ayant son point de vue adapté suivant sa position et son orientation (systèmes CAVE, murs d'écrans, etc...), soit par des dispositifs de taille plus réduite et portable mais à travers lesquels l'utilisateur peut percevoir un monde virtuel à 360 degrés (casques stéréoscopiques, etc...). De plus, la réalité virtuelle offre une perception stéréoscopique de haute qualité indispensable pour une meilleure compréhension des données moléculaires intrinsèquement tridimensionnelles. Même si moins utilisés en biologie structurale, les canaux auditifs et tactiles sont aussi concernés par les développement en réalité virtuelle. L'audio 3d permettant de simuler des sources audio dans un environnement 3d autour de l'utilisateur et les retours haptiques associés à certains dispositifs d'interaction permettent de ressentir davantage l'immersion dans un monde virtuel donné dans la limite de cohérence de leur implémentation. 

Cependant, l'immersion comporte également ses limites. La navigation au sein de dispositifs immersifs dans des données abstraites est un premier obstacle. Découlant de la navigation au sein de scènes virtuelles immersives et également présent pour des scènes réalistes, par opposition aux scènes illustrant des données abstraites, le mal du simulateur est un frein non négligeable à l'expérience de l'utilisateur. Ce phénomène de gène pour l'utilisateur dégrade significativement son expérience et ses performances dans les environnements immersifs. Plusieurs études ont montré que ce mal du simulateur possède plusieurs points communs avec le mal des transports. De la même façon que pour le mal des transports, dans un environnement immersif, le cerveau éprouve une certaine difficulté à dissocier les perceptions visuelles de l'utilisateur à l'implication physique de son corps. Bien souvent, cela est dû à une dissociation de la sollicitation du système vestibulaire par rapport au système perceptif. Cette différence de sollicitation entraîne chez certaines personnes un malaise qui peut aller de légères sensations d'inconfort à de plus importants troubles de l'équilibre ou de nausées. Les paradigmes de navigation en conditions immersives sont multiples mais ont pour la plupart une utilisation ciblée et difficilement généralisables. Ils ont souvent été développé pour répondre à des contraintes de navigation dans des scènes écologiques précises et donc ne parviennent pas à répondre aux attentes d'efficacité dans des scènes virtuelles dont les informations proposées sont de nature différente. Cette absence de compatibilité entre les scènes réalistes et abstraites peut s'expliquer par: 1) une absence de repères spatiaux dans les scènes abstraites où les données scientifiques ne sont généralement pas exposées au sein d'un environnement mais dans un espace vide monochromatique, 2) une absence de notion d'orientation, sans haut ni bas, 3) une exploration centrée sur un objet unique au centre de l'attention. Ces trois points ne peuvent être dissociés de toute mise en place de paradigmes de navigation pour la biologie structurale et les données scientifiques en général.
En parallèle de la navigation, les interactions entre l'utilisateur et son espace de travail dans un environnement immersif sont limitées par l'absence de dispositifs précis comme la combinaison clavier/souris en conditions standards. Afin de garantir une immersion optimale, les menus et commandes textuelles habituelles sont mis de côté au profit d'interactions plus naturelles déclenchées par des interactions directes avec les objets virtuels présentés via des gestes et/ou par des commandes vocales. Ces interactions peuvent également s'appuyer sur certains dispositifs adaptés aux espaces immersifs mais ces derniers comportent un nombre d'entrées limités qui nécessitent d'être également optimisés pour pouvoir déclencher l'ensemble des actions nécessaires au sein d'un espace de travail en biologie structurale. Parmi celles-ci, sélection, changement de rendu, mouvement, accès aux informations sont plusieurs actions importantes en visualisation moléculaire. A ces commandes s'ajoutent toutes les actions propres aux espaces d'analyse et permettant d'interagir avec des graphes. Afin de garantir une optimisation profonde de la couche d'interaction entre l'utilisateur et les espaces de travail, il est donc nécessaire de mettre en place une méthode robuste permettant d'anticiper et de simplifier les étapes d'interaction permettant d'accomplir chaque tâche pouvant être accomplie.

 
\subsection*{Approche générale}

Notre première démarche dans ce travail de thèse fut de fournir un premier contrôle de l'évolution d'une simulation moléculaire à un expert scientifique. Ce contrôle doit permettre d'appréhender la direction vers laquelle tend la simulation et éventuellement mettre en évidence une évolution ne respectant pas les contraintes imposées et l'évolution attendue. Afin de permettre un contrôle sur des données de taille importante et au sein d'une simulation étant en cours d’exécution, nous avons choisi de mettre en place une méthode de capture d'image simple permettant de visualiser sur n'importe quelle plateforme, un aperçu pseudo-3d de la structure étudiée. La légèreté des images utilisées pour ce rendu et le développement axé multi-plateforme de notre application, permet également son utilisation au sein de communications scientifiques diverses (journaux papier, journaux électroniques, actes de conférences, etc...) afin d’accroître la quantité d'information présentée. A la différence des images 2d habituelles, notre méthode permet un rendu pseudo-3d pour des illustrations complexes constituées de plusieurs couches et/ou affichant un objet désirant être affiché en profondeur. Un rendu 3d sous forme de visualiseur 3d est associé à l'application pour des structures entières de molécules dont les structures 3d ont été rendues disponibles en ligne.

Motivé par le manque de paradigmes de navigation spécifiques au domaine de la biologie structural, notre second axe de travail s'est concentré sur le développement de méthodes de navigation basées sur le contenu et la tâche en biologie structurale. Ces méthodes et paradigmes s'inspirent de la nature géométrique de la plupart des complexes moléculaires. Nos développements ne sont néanmoins aucunement contraints à des complexes moléculaires symétriques puisque toute structure particulière retrouvée au sein d'une molécule nous permet de mettre en place nos paradigmes de navigation. Grâce à ces paradigmes, tout au long de son exploration, l'utilisateur 1) garde un point de vue stabilisé, 2) possède des repères spatiaux fixes afin de se situer dans la scènes, 3) peut utiliser des chemins optimaux de déplacement pour atteindre des régions d’intérêt. Ces apports répondent respectivement aux problèmes de malaise pouvant être ressentis à cause: 1) d'une variation trop rapide de l'orientation de l'utilisateur par rapport à sa cible visuel, 2) d'une perte de la situation spatiale de l'utilisateur dans sa scène, 3) d'un trop grand nombre d'étapes ou de temps pour atteindre une région d'intérêt tout en gardant une conscience spatiale de la scène virtuelle. En contraignant la navigation à des chemins optimaux, via une restriction des directions du mouvement et des orientations de la caméra, nous assurons un carcan de possibilités limité mais adapté aux interactions simplifiées prenant place au sein des environnements immersifs. De la même manière, nous anticipons certaines tâches d'exploration en fournissant des paradigmes automatisant certains déplacements considérés comme habituels en exploration moléculaire.

Notre troisième sujet d'études s'intéresse à la possibilité de regrouper les espaces de visualisation et d'analyse afin de raccourcir la boucle d'étude des données de simulation moléculaire et ainsi répondre aux problèmes de surcharge de données. Comme exposé précédemment, dans une situation en temps réel, un enjeu majeur est de réduire les données échangées entre les composants de simulation, de visualisation et d'analyse pour permettre à un utilisateur de contrôler ces trois composants de façon optimale et simultanée à tout moment. Sans une intégration fine de ces composants au sein d'une même plateforme logicielle, il est impossible de garantir un réel impact de ce regroupement dans le processus d'étude. En effet, toute incohérence ou hétérogénéité dans les données échangées entre les modules diminue la généricité de l'approche et la ré-utilisabilité de l'application et potentiellement du contrôle de l'utilisateur sur ces modules de façon synchronisée. Notre approche a donc été de mettre en place une couche d'abstraction supplémentaire décrivant données utilisées par les modules manipulés usuellement par les experts du domaine. Cette couche d'abstraction a pu être mise en place par la construction d'une ontologie et d'une sémantique du domaine. Cette ontologie a pu ensuite être utilisée pour construire une base de données liées, directement inspirée de ce qui peut être fait dans le web sémantique, afin de regrouper l'ensemble des données mises en jeu lors d'une session de travail spécifique et regroupant l'ensemble des informations que possède l'expert scientifique sur son système moléculaire. En restreignant l'utilisation des données de la base de données, définis dans une ontologie précise, aux modules constituant notre application, nous pouvons assurer une homogénéité totale de ce qui est manipulé et donc proposer aux utilisateurs des espaces de visualisation et d'analyse mutualisés qui s'adaptent parfaitement à tout système d’intérêt.


\commentaire{Intro trop courte et trop générale, une intro est un teaser, on doit pouvoir savoir très précisément ce que tu vends, notamment en terme de scénario et de contribution, tu dois détailler chaque point du résumé}

\commentaire{Rappeler rapidement la différence entre temps réel et temps interactif}

\subsection*{Plan du manuscrit}


babla