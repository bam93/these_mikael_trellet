%%%%%%%%%%%%%%%%%%%%%%%%%%%%%%%%%%%%%%%%%%%%%%%%%%%%%%%
%% EN-TETES ET PIEDS DE PAGE
\usepackage{fancyhdr}
\usepackage{color}
\pagestyle{fancy}% pour activer le style de pages personnalis�
\fancyhf{}%remise � z�ro des en-t�te et pied de page
\setlength{\headheight}{28.1pt} % pour fixer la hauteur de l'espace r�serv� � l'en-t�te du haut

%%% Pas de num�ro de page sur la premi�re page des chapitres
\makeatletter
\let\ps@plain=\ps@empty
\makeatother

%===================== Style 1 =================================================
%En-t�te : 
% * dans la boite de droite (R), pour les pages impaires (O)
% * et dans la boite de gauche (L), pour les pages paires (E)
% mettre le num�ro de page (\thepage).
\fancyhead[RO,LE]{% 
\thepage
}
\fancyhead[LO]{\fontsize{9.5}{11} \selectfont \scshape \nouppercase{\rightmark} }  %%%Section
\fancyhead[RE]{\fontsize{9.5}{11} \selectfont \scshape \nouppercase{\leftmark} } %%% Chapitre 
\renewcommand{\headrulewidth}{.4pt}
\fancyfoot{}


%\pagestyle{fancy}     
%\fancyhead{} % clear all header fields
%\fancyhead[L]{\fontsize{10}{12} \selectfont Personal Statement}
%
%\pagestyle{fancy}
%\renewcommand{\chaptermark}[1]{\markboth{#1}{}}
%\renewcommand{\sectionmark}[1]{\markright{\thesection\ #1}}
%\fancyhf{} \fancyhead[LE,RO]{\bfseries\thepage}
%\fancyhead[LO]{\bfseries\rightmark}
%\fancyhead[RE]{\bfseries\leftmark}
%\renewcommand{\headrulewidth}{0.5pt}
%\addtolength{\headheight}{0.5pt}
%\renewcommand{\footrulewidth}{0pt}
%\fancypagestyle{plain}{ \fancyhead{}
%\renewcommand{\headrulewidth}{0pt}} 

%================================== Style 2 ====================================

% \fancyfoot[RO,LE]{% Boite de droite (R), pages impaires(O) et Boites de gauche pages paires
% \thepage
% }
% \fancyhead[CO]{\slshape \nouppercase{\rightmark}}  %%%Section
% \fancyhead[CE]{\slshape \nouppercase{\leftmark}} %%% Chapitre 
% \renewcommand{\headrulewidth}{.4pt}

% Remarques generales :
% nouppercase permet l'affichage en minuscules au lieu de majuscules
% slshape permet l'affichage en lettres pench�s
% scshape permet l'affichage en petites capitales

% Pour que les pages paires sans texte (par exemple, � la fin d'un chapitre et
% avant un autre), ne contiennent ni en-t�te ni pied de page (source :
% http://www.tex.ac.uk/cgi-bin/texfaq2html?label=reallyblank)
\let\origdoublepage\cleardoublepage
\newcommand{\clearemptydoublepage}{%
  \clearpage
  {\pagestyle{empty}\origdoublepage}%
}
\let\cleardoublepage\clearemptydoublepage

% R�glage fin des notes de bas de page
\FrenchFootnotes % pour les notes de bas de page � la fran�aise
\AddThinSpaceBeforeFootnotes % pour avoir une espace fine entre le mot et l'appel de note


%%%%%%%%%%%%%%%%%%%%%%%%%%%%%%%%%%%%%%%%%%%%%%%%%%%%%%%
%% CHAPITRE ETOILE
%% avec r�f�rence dans la table des mati�res et les bons en-t�tes
%% il sert pour l'introduction, la page de notations.
\newcommand*\chapterstar[1]{%
  \chapter*{#1}%
%  \addcontentsline{toc}{chapter}{#1}%
 \addstarredchapter{#1} 
  \markboth{#1}{#1}}

%%%%%%%%%%%%%%%%%%%%%%%%%%%%%%%%%%%%%%%%%%%%%%%%%%%%%%%
%% PARTIE ETOILE
%% avec r�f�rence dans la table des mati�res et les bons en-t�tes
%% il sert pour l'introduction, la page de notations.
\newcommand*\partstar[1]{%
  \part*{#1}%
%  \addcontentsline{toc}{chapter}{#1}%
 \addstarredpart{#1} 
  \markboth{#1}{#1}}


%%%%%%%%%%%%%%%%%%%%%%%%%%%%%%%%%%%%%%%%%%%%%%%%%%%%%%%%
% ENVIRONNEMENTS DE THEOREMES
\theoremstyle{plain} % style plain
\newtheorem{theo}{Théorème}[chapter]
\newtheorem{cor}[theo]{Corollaire}
\newtheorem{prop}[theo]{Proposition}
\newtheorem{lem}[theo]{Lemme}
\newtheorem{conj}[theo]{Conjecture}
\newtheorem*{theoetoile}{Théorème} % th�or�me non num�rot�
\newtheorem*{conjetoile}{Conjecture} % conjecture non num�rot�e

\theoremstyle{definition} % style definition
\newtheorem{defi}[theo]{Définition}
\newtheorem{exemple}[theo]{Exemple}
\newtheorem{question}[theo]{Question}
\newtheorem{remarque}[theo]{Remarque}
\newtheorem{notation}[theo]{Notation}

% Pour renommer ``preuve'' en ``d�monstration''
\renewcommand{\proofname}{Démonstration}


%%%%%%%%%%%%%%%%%%%%%%%%%%%%%%%%%%%%%%%%%%%%%%%%%%%%%%%
% ENVIRONNEMENTS DEDICACE ET EPIGRAPHE
\newenvironment{dedicace}{%
  \newpage\thispagestyle{empty}
  \hfill\begin{minipage}{100mm}\begin{flushright}\it}{%
  \end{flushright}\end{minipage}\vfill}

\newenvironment{epigraphe}{%
  \hfill\begin{minipage}{60mm}\begin{flushright}\footnotesize\it}{%
  \end{flushright}\end{minipage}\hspace*{7mm}\vfill}

\newcommand{\commentaire}[1]{\textcolor{red}{#1}}
\newcommand{\style}{\textcolor{red}{style à revoir}}

\newcommand{\correction}[1]{\textcolor{blue}{#1}}

