%!TEX root = these.tex

%\chapterstar{CONCLUSION GÉNÉRALE ET PERSPECTIVES}


\chapter*{Conclusion générale et perspectives}
\mtcaddchapter

\addstarredpart{CONCLUSION GÉNÉRALE ET PERSPECTIVES} % Pour ajouter une partie fictive dans la table des matière
\mtcaddpart

\markboth{Conclusion générale et perspectives}{Conclusion générale et perspectives}



\section*{Contributions de la thèse}


\subsection*{Application de communication scientifique immersive}

Les moyens de communication dans le monde scientifique ont connu une évolution rapide ces dernières années profitant de l'évolution des moyens de communication standard. Ces derniers sont maintenant essentiellement basés autour du réseau internet et du support informatique. Au-delà de la simple lecture d'articles scientifiques sur un écran fixe, on tend à déporter cette lecture sur des périphériques mobiles type smartphones ou tablettes, ces derniers possédant la résolution nécessaire pour une expérience de lecture satisfaisante. Les moyens d'interactions avec les informations scientifiques ont également changé, profitant de l'outil informatique pour ajouter des couches de données aux simples textes et images.

Notre approche s'est inscrite dans cette évolution et avait pour but de fournir une méthode simple et rapide pour explorer des données de structures 3d de molécules à partir de fichiers légers et facilement échangeables ou téléchargeables. Basée sur un moteur de jeu largement utilisé au sein de la communauté des développeurs d'application mobiles, Unity3D, nous avons créé une application permettant de générer un objet 3d correspondant à une scène moléculaire grâce à l'utilisation de seulement 2 fichiers images, une image de texture et une carte de profondeur. Grâce à l'utilisation du gyroscope présent dans la grande majorité des périphériques mobiles, nous pouvons permettre à l'utilisateur de visualiser sous plusieurs angles une structure moléculaire avec la perception de profondeur qu'offre la 3d.
Dans la continuité de notre approche, nous avons également mis au point la possibilité d'explorer un modèle 3d complet de molécule en situant l'utilisateur au milieu de sa structure et permettant à l'utilisateur d'utiliser son smartphone comme une fenêtre sur le monde virtuel.

\subsection*{Navigation guidée par le contenu et la tâche}

L'évolution de la Réalité Virtuelle est passée à la fois par l'évolution de ses techniques d'immersion (affichage, audio), d'interactions (périphériques, retours sensoriels, gestes) mais également de navigation (suivis de mouvement, périphériques fixes). 
Nous avons cependant mis en avant une certaine orientation des évolution de la navigation vers l'exploration de scènes virtuelles réalistes dont les repères spatiaux et les rappels du réels sont nombreux. La visualisation et l'exploration de données scientifiques souffre d'un certain retard dans l'adaptation des paradigmes de navigation les concernant. Il est possible d'expliquer ce retard par le retard pris par la RV pour être appliquée aux finalités scientifiques. Les développements récents ont pourtant réduit ce retard et de nombreux domaines scientifiques utilisent désormais des EV immersifs pour répondre à certaines problématiques. L'apprentissage et la visualisation de données complexes ont pu particulièrement profiter de l'apport de la perception de profondeur et la mise en place de techniques d'interactions proches des outils et instruments réels.

Fort du constat de l'absence de paradigmes spécifiques pour l'exploration moléculaire, nous avons proposé plusieurs paradigmes de navigation basés sur le contenu moléculaire et les tâches que les experts scientifiques peuvent être amené à effectuer dans des EV immersifs. Ces paradigmes répondent aux différentes échelles de granularité que présente la visualisation de données scientifiques complexes, à savoir une première approche d'exploration globale avant un focus sur des phénomènes locaux et détaillés. Notre développement a pris en compte au maximum la problématique du malaise du virtuel que nous avons mis en avant comme très présent lors de l'exploration de données abstraites comme peuvent l'être les complexes moléculaires.

\subsection*{Simplifier la fusion entre visualisation et analyses de données complexes}

La boucle illustrant le processus standard de l'étude d'un phénomène biologique à l'échelle de la structure de ses acteurs présente une certaine dissociation des étapes de visualisation et d'analyses pourtant toutes deux très proches et présentant de nombreuses passerelles d'échanges de données. Conscients de la nécessité d'optimiser les processus d'interactions afin de coller au plus près des standards imposés par la RV et profitant de la capacité d'immersion qu'elle fournit, nous avons cherché à mettre en place un cadre logiciel générique permettant de lier visualisation et analyses au sein d'un même espace de travail.

Cette fusion des activités n'a pu se faire que grâce à la mise en place d'une description haut-niveau des concepts mis en jeu dans ces deux activités. Nous nous sommes inspirés des récentes avancées faites dans le Web Sémantique pour parvenir à créer une ontologie métier spécialisée nous permettant de structurer les différents concepts d'analyses et de visualisation et de les lier autour d'une même problématique, la visualisation analytique structurale. 
Grâce aux outils du web sémantique, il nous est possible de stocker autour d'une architecture définie l'ensemble des données générées au cours d'un processus d'étude de biologie structurale. Au-delà de la simplicité d'accès à ces données, il est possible de raisonner sur celles-ci et de mettre en place des liens forts entre leurs différentes représentations graphiques. Il est ainsi aisé de créer des méthodes d'interactions simplifiées permettant de passer de la visualisation à l'analyse et inversement dans un temps interactif.


\section*{Perspectives}

\subsection{Immersion cognitive via périphériques mobiles}

\subsubsection*{Intégration au sein d'un environnement d'utilisation dédié}

L'une des forces de notre application DepthMol3D est sa simplicité d'utilisation couplée à son absence de contraintes matérielles et logicielles. Dans le monde connecté d'aujourd'hui, il est important de permettre un accès aux données à communiquer à tout instant. L'utilisation de technologies de QR-codes est une piste crédible pour permettre d'accélérer le processus d'acquisition des images nécessaires à l'application. Grâce à des méthodes de stockage dans le \textit{cloud}, il est possible d'imaginer la possibilité de rendre disponible à tout instant un ensemble de cartes de profondeur et leurs textures associées pouvant être téléchargées et chargées par l'application suite à la simple détection d'un QR-code (voir Figure \ref{Fig:qr_code}).

\begin{figure}[h]
  \centering
  {\includegraphics[width=.75\linewidth]{./figures/conclusion/qr_code.jpg}}
    \caption{{\it Exemple de QR-code redirigeant vers un site web.}}
  \label{Fig:qr_code}
  \hspace{0.2cm}
\end{figure}

\subsection*{Amélioration des repères spatiaux et élargissement des tâches expertes supportées}

Le retour d'expérience des utilisateurs d'un logiciel est important mais néanmoins souvent oublié quand on arrive aux étapes de post-développement. Avec l'évolution rapide des techniques, plateformes et approches de visualisation, il est important de revoir les nouveaux besoins des experts et les nouvelles possibilités offertes par les nouvelles technologies. 

Il nous semble également important d'utiliser l'expertise scientifique afin de créer des paradigmes de navigation répondant à des besoins spécifiques, ceux-ci évoluant avec la nature du complexe moléculaire étudié. Ce plus haut niveau d'interaction ne doit pourtant pas dépasser le niveau d'implication cognitive que doit comporter la navigation, cette dernière restant bien souvent un moyen d'accéder à d'autres tâches plus complexes et nécessitant davantage de réflexion de la part de l'utilisateur. Et même si les études ont montré qu'une augmentation de la charge cognitive est une bonne manière de réduire le \textit{cybersickness} (voir section \ref{cybersickness}) elle peut également réduire l'efficacité des tâches annexes au processus de navigation.

\subsection*{Visualisation analytique et sémantique}

\subsubsection{Multimodalité pour l'interaction vocale}

\subsubsection{Automatisation de la génération de graphes}

\section*{Remerciements spécifiques}

“This work was conducted using the Protégé resource, which is supported by grant GM10331601 from the National Institute of General Medical Sciences of the United States National Institutes of Health.” 