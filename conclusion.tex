%!TEX root = these.tex

%\chapterstar{CONCLUSION GÉNÉRALE ET PERSPECTIVES}


\chapter*{Conclusion générale et perspectives}
\mtcaddchapter

\addstarredpart{CONCLUSION GÉNÉRALE ET PERSPECTIVES} % Pour ajouter une partie fictive dans la table des matière
\mtcaddpart

\markboth{Conclusion générale et perspectives}{Conclusion générale et perspectives}

En réponse à la complexité grandissantes des modèles 3d de structures moléculaires et à l'augmentation de la quantité de données générées au cours du processus d'étude de ces mêmes structures, nous proposons une nouvelle approche qui permet de résoudre ces problématique tout en plaçant l'expert scientifique au sein d'une plateforme de travail immersive regroupant l'ensemble des données mises en jeu dans le processus d'étude. Après avoir identifié les besoins et limites d'une telle approche, nous avons développé plusieurs outils dédiés à franchir ce pas. 

Nous avons tout d'abord développé des paradigmes de navigation basés sur le contenu moléculaire et la tâche des experts scientifiques pour l'exploration de structures 3d moléculaires complexes au sein de dispositifs immersifs. Nous avons ensuite permis l'intégration du contexte de visualisation et du contexte  d'analyses au sein d'un même contexte interactif grâce une formalisation sémantique opérationnelle et efficace. Pour évaluer ces dernier paradigmes, notamment en terme d'évaluation, nous avons proposé une méthodologie théorique d'évaluation basée sur l'analyse ergonomique des tache puis, la modélisation de ces tâches métier complexe par une décomposition hiérarchique, les taches atomique étant celles dont on est capable d'estimer le temps de complétion. 

\section*{Contributions de la thèse}

Nous avons mis en évidence, dans la première partie de ce manuscrit, les \textbf{usages en biologie structurale}, coeur d'application de nos travaux de recherche. Ce domaine de la biologie moléculaire, cherchant à décrypter les structures des protéines afin d'en déduire leurs fonctions, a su évoluer avec les outils informatiques, en dépit de l'importance des méthodes expérimentales dans la résolution de structures 3d. 
L'informatique a tout d'abord fait partie intégrante du traitement des signaux générés par la majorité des méthodes expérimentales pour être ensuite considérée comme support de méthodes fiables de modélisation de structures protéiques. La modélisation théorique de structures moléculaire reçu d'ailleurs ses plus belles lettres de noblesse en 2013, ses pionniers recevant le prix Nobel de chimie. Dans la continuité, l'informatique graphique a permis l'évolution des méthodes de communication et de diffusion des modèles de structures dans le monde scientifique, permettant de partager les avancées du domaine au plus grand nombre.

Parmi les nombreuses approches théoriques dédiées à la compréhension de la structuration des biomolécules du vivant, la \textbf{visualisation moléculaire} fut un pilier de la production et la transmission de connaissances, et ce depuis le début de la biologie structurale dans les années 50. Comme pour tout phénomène structurel à observer, il est naturel que le processus d'analyse passe par l'observation de celui-ci, aucun indicateur quantitatif ne décrivant une particularité géométrique mieux que sa visualisation directe. Elle rencontre cependant ses limites avec la complexité croissante des modèles structuraux à observer. La taille des complexes moléculaires oblige à repenser les méthodes d'exploration usuelles, introduites par les logiciels de visualisation experts et jusqu'ici basées autour du concept de manipulation des représentations de molécules.


%%%%%%%%%%%%% RV %%%%%%%%%%%%%%%%%%

Par ailleurs, ces 10 dernières années ont été le témoin de l'essor de la \textbf{Réalité Virtuelle} comme outil de choix pour immerger l'utilisateur au coeur de ses expériences, qu'elles soient ludiques ou professionnelles. Elle offre la possibilité de créer un espace de visualisation 3d et à 360 degrés où l'expert peut s'immerger avec un degré d'immersion visuelle afin d'explorer plus naturellement, et sans limitations d'espace, ses données. Nous avons donc dessiné dans une seconde partie les contours de la RV, ses concepts, ses supports technologiques ainsi que son historique, passant à la fois par l'évolution de ses techniques d'immersion (affichage, audio), d'interactions (périphériques, retours sensoriels, gestes) mais également de navigation (suivis de mouvement, périphériques physiques). 

Nous avons cependant mis en avant une orientation nette des évolution de la navigation vers l'exploration de scènes virtuelles réalistes dont les repères spatiaux et les rappels du réels sont nombreux. La visualisation et l'exploration de données scientifiques souffrent d'un retard dans l'adaptation des paradigmes de navigation les concernant au regard de l'avancée des paradigmes de navigation proposés dans le domaines de la RV.
Il contraste avec les développements récents qui ont permis à de nombreux domaines scientifiques d'utiliser désormais des EV immersifs pour répondre à leurs problématiques. La visualisation de données complexes a pu par exemple profiter de l'apport de la perception de profondeur et la mise en place de techniques d'interactions proches des outils et instruments du monde réel. Usuellement articulée autour de la manipulation, nous avons également souligné que la navigation en conditions immersive ne peut se faire de la même façon. 


%%%%%%%%%%%% Navigation guidée %%%%%%%%%%%%%

Fort du constat de l'absence de paradigmes pour l'exploration de structures moléculaire et plus généralement de données scientifiques abstraites dans des dispositifs immersifs, nous avons proposé dans une troisième partie plusieurs paradigmes de navigation basés sur le contenu moléculaire et les tâches que les experts scientifiques sont amené à effectuer. Ces paradigmes répondent aux différentes échelles de granularité présentes en visualisation scientifique, à savoir une première approche d'exploration globale avant un focus sur des phénomènes locaux et détaillés. Notre développement a particulièrement attentif la problématique du mal du simulateur, mis en avant lors de l'exploration de données abstraites comme peuvent l'être les complexes moléculaires. Nous avons tiré parti de la particularité géométrique commune des complexes moléculaire de grande taille, leur agencement symétrique, pour alimenter la conscience spatiale de l'utilisateur.

Grâce à l'utilisation d'axes et centres de symétries, nos paradigmes de navigation contraignent l'utilisateur autour de chemins de navigation adaptés pour l'exploration de structures moléculaires de grande taille. Nous avons considérés des chemins de navigation externes et internes aux complexes et mis en place des solutions de génération automatique de chemins préférentiels dans le cadre de tâches expertes spécifiques comme l'accès à des régions d'intérêt enfouies dans le complexe ou la comparaison de phénomènes biologiques répétés sur les différentes sous-unités constituant le complexe moléculaire.
Nos paradigmes, centrés sur le contenu moléculaire et la spécificité symétrique du complexe observé ainsi que sur les tâches expertes en biologie structurale, sont indépendants du contexte d'interaction. Ils répondent à un manque concret de paradigmes de navigation dédiés à l'exploration de données abstraites dans des contextes immersifs mais sont adaptables aux environnement de travail plus classiques. 

%%%%%%%%%%% Appli mobile %%%%%%%%%%%%%%%%%

% Motivés par les apports de la RV dans les technologies de visualisation avancées et l'approche d'immersion cognitive, notre intérêt s'est ensuite porté sur les moyens de communication dans le monde scientifique. Ces derniers ont connu une évolution rapide ces dernières années profitant de l'évolution des moyens de communication standards. Ces derniers sont maintenant essentiellement basés autour du réseau internet et du support informatique. Au-delà de la simple lecture d'articles scientifiques sur un écran fixe, on tend à déporter cette lecture sur des périphériques mobiles type smartphones ou tablettes, ces derniers possédant la résolution nécessaire pour une expérience de lecture satisfaisante. Les moyens d'interactions avec les informations scientifiques ont également changé, profitant de l'outil informatique pour ajouter des couches de données aux simples textes et images.

% Notre approche s'est inscrite dans cette évolution et a pour but de fournir une méthode simple et rapide pour explorer des données de structures 3d de molécules à partir de fichiers légers et facilement échangeables ou téléchargeables. Basée sur un moteur de jeu largement utilisé au sein de la communauté des développeurs d'application mobiles, Unity3D, nous avons créé une application permettant de générer un objet 3d correspondant à une scène moléculaire grâce à l'utilisation de seulement 2 fichiers images, une image de texture et une carte de profondeur. Grâce à l'utilisation du gyroscope présent dans la grande majorité des périphériques mobiles, nous pouvons permettre à l'utilisateur de visualiser sous plusieurs angles une structure moléculaire avec la perception de profondeur qu'offre la 3d.
% Dans la continuité de notre approche, nous avons également mis au point la possibilité d'explorer un modèle 3d complet de molécule en situant l'utilisateur au milieu de sa structure et permettant à l'utilisateur d'utiliser son smartphone comme une fenêtre sur le monde virtuel.
% L'une des forces de notre application DepthMol3D est sa simplicité d'utilisation couplée à son absence de contraintes matérielles et logicielles. Nous fournissons plusieurs exemples de tutoriels et scripts destinés à générer des rendus 2d de profondeur et de texture au sein d'applications de visualisation expertes comme PyMol, VMD et Yasara.

%%%%%%%%%%% Visu Ana %%%%%%%%%%%%%%%%

La quatrième partie du manuscrit réintroduit la volonté de mettre l'utilisateur au coeur de la visualisation et de l'analyse du processus d'étude grâce à des concepts de \textit{Visual Analytics} afin de lui permettre de filtrer les données générées et ainsi réduire la quantité de données rapatriée depuis les centres de calcul vers les laboratoires. Nous sommes revenus sur la nécessité d'intégrer un tel schéma de travail afin de faire évoluer le schéma actuel, passant aujourd'hui par une dissociation des étapes de visualisation et d'analyses, pourtant tout deux étroitement liées.
Le rapprochement de ces deux espaces est l'un des sujets d'étude central du \textit{Visual Analytics}. Ce domaine se base sur la mise en place d'une interactivité forte au coeur de la visualisation et de l'analyse de données scientifiques. Notre réflexion autour des techniques de \textit{Visual Analytics} nous ont amené à considéré comme solution la plus adaptée, la réunion des concepts manipulés au sein des espaces de visualisation et d'analyses.
Nous avons montré que grâce aux outils du web sémantique, il est possible de stocker l'ensemble des données générées au cours d'un processus d'étude de biologie structurale. Au-delà de la simplicité d'accès à ces données, il est possible de raisonner sur celles-ci et de mettre en place des liens forts entre leurs différentes représentations graphiques. La création de méthodes d'interactions simplifiées est simplifiée et permet de passer de la visualisation à l'analyse et inversement dans un temps interactif.

Cette fusion des activités n'a pu se faire que grâce à la mise en place d'une description haut-niveau des concepts mis en jeu dans ces deux activités. La cinquième et dernière partie de notre manuscrit met en avant l'utilisation des récentes avancées faites dans le Web Sémantique pour parvenir à créer une ontologie métier spécialisée nous permettant de structurer les différents concepts d'analyses et de visualisation et de les lier autour d'une même problématique, la visualisation analytique structurale.
Nous avons donc détaillé la création d'une plateforme de travail mixant visualisation et analyses au sein d'un même espace de travail immersif. Après avoir introduit les différentes facettes de notre plateforme, nous l'avons évalué au moyen de scénario précis et de méthodes ergonomiques existantes.

Conscients de la nécessité d'optimiser les processus d'interactions afin de coller au plus près des standards imposés par la RV et profitant de la capacité d'immersion qu'elle fournit, nous avons cherché à mettre en place un cadre logiciel générique permettant de lier visualisation et analyses au sein d'un même espace de travail.


\section*{Perspectives}

% \subsection{Immersion cognitive via périphériques mobiles}

% Dans le monde connecté d'aujourd'hui, il est important de permettre un accès aux données à communiquer à tout instant. L'utilisation de technologies de QR-codes est une piste crédible pour permettre d'accélérer le processus d'acquisition des images nécessaires à l'application. Grâce à des méthodes de stockage dans le \textit{cloud}, il est possible d'imaginer la possibilité de rendre disponible à tout instant un ensemble de cartes de profondeur et leurs textures associées pouvant être téléchargées et chargées par l'application suite à la simple détection d'un QR-code (voir Figure \ref{Fig:qr_code}).

% \begin{figure}[h]
%   \centering
%   {\includegraphics[width=.75\linewidth]{./figures/conclusion/qr_code.jpg}}
%     \caption{{\it Exemple de QR-code redirigeant vers un site web.}}
%   \label{Fig:qr_code}
%   \hspace{0.2cm}
% \end{figure}

% Les évolutions de notre application vont également passer par les usages. Grâce à un paramétrage des conditions de déclenchement des rendus 2d lors d'une expérience de simulation, il est possible de mettre en place un système automatisé de contrôle générant des captures de l'évolution du système à intervalles de temps régulier. L'extrême légèreté des images nécessaires à l'application (quelques kilo-octets pour la carte de profondeur et moins de 1 méga-octet pour la texture) permet de générer et envoyer les images de façon systématique sans risques de surcharge de stockage.

% La résolution des cartes de profondeur est actuellement limitée à 250*250 pixels, résolution basse et ne permettant pas toujours de rapporter au mieux les différences de profondeur entre certaines zones fines d'une scène. Les contours d'atomes ou les liens sous forme de lignes peuvent parfois ne pas être intégrés dans la carte de profondeur générée, demandant ainsi un travail de traitement de l'image pour parvenir à faire ressortir les éléments concernés. La résolution maximum imposée découle de la limitation technique de Unity3D pour générer un objet 3d unique constitué de plus de 65 000 sommets. Notre projet à court terme est le contournement de cette limite, permettant ainsi d'accepter des cartes de profondeur de plus grande résolution et donc d'avoir une perception 3d de la scène encore plus importante.

\subsection*{Amélioration des repères spatiaux et élargissement des tâches expertes supportées}

Le retour d'expérience des utilisateurs d'un logiciel est important mais néanmoins souvent oublié quand on arrive aux étapes de post-développement. Avec l'évolution rapide des techniques, plateformes et approches de visualisation, il est important de revoir les nouveaux besoins des experts et les nouvelles possibilités offertes par les nouvelles technologies. 

Il nous semble également important, d'un point de vue applicatif, d'utiliser l'expertise de l'utilisateur pour qu'il décide des motifs et structures qu'il considère comme importants afin qu'ils constituent la base de la génération automatique de chemins de navigation. Cette possibilité prendrait en compte la nature différente des complexes moléculaires observés mais également le désir de l'utilisateur de mettre l'accent sur une particularité structurelle autre que l'axe de symétrie ou principal d'un complexe.
Les implémentations d'interactions pour l'utilisateur se situent actuellement dans le choix de cet axe principal au commencement de son expérience de navigation et le choix d'une cible pour la recherche d'un point de vue optimal, possible tout au long de la navigation. 

Il est nécessaire d'équilibrer le nombre de processus d'interactions prenant place pendant la navigation, cette dernière restant bien souvent un moyen d'accéder à d'autres tâches plus complexes, nécessitant davantage de réflexion de la part de l'utilisateur. Et même si les études ont montré qu'une implication plus importante de l'utilisateur dans sa tâche de navigation est une bonne manière de réduire le \textit{cybersickness} (voir section \ref{cybersickness}) elle peut également réduire l'efficacité des tâches annexes au processus de navigation.

Les premières tâches expertes considérées comme base pour nos paradigmes ne constituent pas une liste exhaustive des tâches pouvant être effectuées pendant une session d'exploration moléculaires. Parmi les tâches supplémentaires qui pourraient profiter de chemins de navigation créés au moyen de bases géométriques, le parcours de surfaces ou de structures moléculaires de grande taille (comme les membranes cellulaires) pourrait constituer une aide précieuse pour la recherche manuelle de singularités (poche de liaison, déformation locale, etc.). Le calcul de chemins de navigation pourrait s'effectuer à l'aide des algorithmes de rendus surfaciques, ces algorithmes parcourant l'ensemble des atomes pour générer une surface. Des calculs d'extrapolation de positions de molécules peuvent également se faire pour le suivi des membranes.

L'axe de développement autour de la réduction du \textit{cybersickness} est passée par l'ajout, quelque peu empirique, de repères spatiaux, directs comme la \textit{skybox} orientée, ou indirects, comme la contrainte d'orientation tout au long de l'exploration, mais pourrait aller plus loin. La conception de scènes d'immersion réalistes utilisent par défaut des repères spatiaux universels qui constituent, sans aucune intrusion superficielle et pouvant dégrader la sensation d'immersion, des moyens de s'orienter naturellement. La connaissance de la position du soleil est par exemple un bon moyen de savoir à chaque instant notre orientation au sein d'une scène virtuelle.
Il serait aussi possible d'envisager une sonification de la scène, facteur supplémentaire d'immersion. La spatialisation sonore, si correctement effectuée, à travers l'identification et le positionnement de sources sonores en accord avec le contenu, peut apporter un nouveau moyen d'orientation.
 

\subsection*{Visualisation analytique et sémantique}

\subsubsection*{Multimodalité pour l'interaction vocale}

La mise en place d'une ontologie pour définir l'ensemble des concepts mis en jeu au sein de notre plateforme, nous a permis de construire un moteur de conversion qui, couplé à une reconnaissance vocale, permet de convertir un ensemble de mots-clés, en une commande complète et interprétable par le logiciel de visualisation moléculaire utilisé. 

Il est possible d'aller plus loin dans l'interprétation de concepts clés pour la mise en place d'actions ou de commandes logicielles. Il serait en effet possible de profiter de la connaissance métier du moteur pour interpréter et enregistrer des interactions de l'utilisateur avec ses espaces d'analyses ou de visualisation et d'appliquer des commandes vocales aux objets cibles des interactions de l'utilisateur. Ainsi, dans un contexte de visualisation moléculaire, si l'utilisateur sélectionne un résidu et commande vocalement l'action de coloration en bleu, le moteur multimodale devra être capable de savoir que l'action de coloration en bleu doit être appliquée au résidu sélectionné. 

Si nous regardons la définition de l'interaction multimodale, deux niveaux peuvent profiter de l'implémentation de notre approche par représentation de connaissances :

\begin{itemize}
	\item La gestion des événements périphériques en entrée. Chaque 
	\item L'intégration du type de contenu / L'observation d'événements
\end{itemize}

Les suites logiques de notre travail concernent l'utilisation de la sémantique par le formalisme RDF/RDFS/OWL pour construire des solutions de commandes vocales opérationnelles, du fait de sa simplicité pour l'expert (suite de mot clé dans la terminologie du domaine), d'une forte tolérance au désordre, concernent la supervision de la multimodalité en général.
 
En effet, notre approche intègre la connaissance du moléculaire, des tâches métier relatives à la manipulation de ce contenu, tout en formalisant dans une représentation homogène tous les événements d'interactions génériques donc indépendants de l'application et du domaine ciblé. Les observeurs de contexte interactif dans des architectures de supervision de la multimodalité en entrée [martin 2012 SPIE], en produiraient des fait et des événements interactifs formalisées de manière sémantique de tous les événements d'interaction de l'utilisateur dans le même formalisme que celui utilisé pour modéliser la connaissance. Ensuite, le haut niveau de performance du langage SPARQL permet d'envisager d'effectuer des requêtes à la fois sur les contenus manipulés et sur la nature des interactions, permettant de construire et de déclencher des commandes multimodales appropriées au contenu manipulé (nature de l'objet, atome) et à la nature de l'interaction (focus, pointage, navigation, sélection, commande vocale). Le formalisme RDFS/RDF/OWL et le langage SPARQL permet d'énoncer par ailleurs des règles d'inférences essentielles à la construction de ces commandes multimodales, pour répondre en particulier aux problématiques de la multimodalité dans un contexte collaboratif. 
Dans un tel contexte, deux utilisateurs peuvent chacun émettre une commande multimodale de manière conjointe, qu'il peut être difficile à interpréter sans règle, si les règles sont incohérentes, ou si elles provoquent une collision sur un contenu partagé. Il s'agira donc d'intégrer des règles, dans un futur superviseur de la multimodalité en entrée, basé sur ce formalisme, prenant en compte le fait que pour certains auteurs, un utilisateur dans un environnement collaboratif durant l'interaction multimodale doit être considéré comme une modalité [martin 2012 SPIE],.



\subsubsection*{Automatisation de la génération de graphes}

Le processus de création de représentations d'analyses sous-jacent à notre plateforme est actuellement entièrement dépendant des paramètres de visualisation codés au sein des scripts javascript utilisés. Les nuages de point sont par défaut le moyen de représentation utilisé pour visualiser les valeurs numériques contenues dans la base de données RDF. Les représentations analytiques peuvent cependant prendre d'autres formes, variantes suivant la nature des données à représentées et l'information désirant en être extraites. 
La fréquence d'apparition, au sein de l'ensemble des modèles d'une simulation, de l'association de valeurs de deux propriétés distinctes (polarité et charge par exemple) pourra être représenté grâce à une matrice dont le nombre de colonnes correspondra au nombre de valeurs que peut prendre la propriété 1, le nombre de lignes correspondra au nombre de valeurs que peut prendre la propriété 2. Les fréquences d'apparition seront elles représentées par un gradient de couleur codant pour des rangs de valeurs prédéfinis.
La connaissance du type et de la nature des données en entrée ainsi que le type et la nature du type de donnée en sortie, permet de restreindre les possibilités de représentation à un ou deux formats précis. Cette connaissance des types et natures de données est justement présente au sein de l'ontologie mise en place et pourrait donc permettre, selon un simple choix des données à afficher de la part de l'utilisateur, de générer une représentation adaptée de façon totalement automatisée. 
Une étape supplémentaire pourrait considérer le choix des données d'entrée non plus comme choix seul de l'utilisateur, mais au sein d'un processus automatisé conséquent à une interaction spécifique dans l'espace de visualisation. La nature et le type des données d'entrée, reconnus par l'ontologie, filtreraient la liste des informations de sorties pouvant être extraites de leur mise en relation dans l'espace d'analyses et ainsi aboutir à une liste réduite de moyens de représentations.

\section*{Bilan global}

Nos approches ouvrent la porte à une nouvelle génération d'applications en biologie structurale, largement inspirées des évolutions d'utilisation des technologie d'information, de visualisation et de traitement des données du monde d'aujourd'hui. L'essor de la RV n'est plus que théorique et cantonnée aux laboratoire de recherche, elle se démocratise et avec elle de nombreux supports immersifs voient le jour pour intégrer l'utilisateur au coeur d'un monde de données virtuelles liées. Nous avons essayé de fournir dans ce travail de thèse quelques solutions aux limites actuelles existantes pour une intégration complète des outils de biologie structurale au sein de dispositifs immersifs. Ce travail fait le premier pas vers une plateforme complète où les espaces de visualisation de structures 3d côtoient les informations analytiques pertinentes pour l'utilisateur tout au long de sa session de travail. Motivé par la dimension immersive, nos contributions n'en sont pas moins applicables aux stations de travail habituelles et nous sommes parvenus à mettre en place des solutions génériques, multi-plateformes et indépendantes du contexte.
