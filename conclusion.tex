%\chapterstar{CONCLUSION GÉNÉRALE ET PERSPECTIVES}


\chapter*{Conclusion générale et perspectives}
\mtcaddchapter

\addstarredpart{CONCLUSION GÉNÉRALE ET PERSPECTIVES} % Pour ajouter une partie fictive dans la table des matière
\mtcaddpart

\markboth{Conclusion générale et perspectives}{Conclusion générale et perspectives}



\section*{Contributions de la thèse}

\subsubsection*{Développement d'instruments de synthèse vocale}

\lipsum[1-2]


\subsection*{Contrôle chironomique de l'intonation musicale}
\lipsum[1-2]

\subsubsection*{Gestes instrumentaux et jeux collectifs}
\lipsum[1-2]

\subsubsection*{Application pédagogique}
\lipsum[1-2]

\section*{Perspectives}

\subsubsection*{Vers de nouveaux instruments de voix numériques}
\lipsum[1-2]

\subsubsection*{Comparaison de tâches articulatoires vocales et chironomiques}
\lipsum[1-2]

\subsubsection*{Étude du Chorus Digitalis}
\lipsum[1-2]
