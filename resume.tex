%\skiptoevenpage 
%\addcontentsline{toc}{chapter}{Résumé / Abstract}
\backmatter 
\chapter*{}% si on enlève le \chapter , le lien dans le pdf revoit vers la biblio 
\pagestyle{empty}

\setlength{\headheight}{0.pt}
% \thispagestyle{empty} 
%\pdfbookmark[0]{Résumé}{resume}   
%~ % pour que la page soit la bonne dans la table des matière 
        
\includegraphics[height=1.cm]{./figures/UPMC_these.pdf}\hfill
\includegraphics[height=1.cm]{./figures/limsilogo_vectoriel.pdf}\hfill


\begin{center}
 \textbf{Mikael Trellet} \\
 \textbf{EXPLORATION ET ANALYSE INTERACTIVE DE DONNÉES MOLÉCULAIRES EN CONDITION IMMERSIVE}
\end{center}
    
%\fontsize{10}{12}
%\selectfont

\footnotesize
\subsection*{Résumé}

%\tiny
%(minuscule)

%\scriptsize
%(très petit)

%\footnotesize
%(assez petit)

%\small
%(petit)

%\normalsize
% normal

%\large
%(grand)

%\Large
%(plus grand)

%\LARGE
%(très grand)

%\huge
%(énorme)



\footnotesize
En biologie structurale, l'étude de structures moléculaires comporte quatre grandes activités : la collecte de résultats expérimentaux ou théoriques, la visualisation des structures, la simulation moléculaire, l'analyse et l'interprétation des résultats. Chacune de ces activités s'organise autour d'une boucle séquentielle, dans des contextes de travail interactifs et applicatifs indépendants, générant de nouvelles hypothèses à vérifier par l'expérimentation.

La quantité de données mise en jeu dans cette boucle pose désormais un double problème: sa production de plus en plus massive fait exploser le temps passé par le chercheur dans des taches de transfert et de conversion imposées par les différents outils et ressources propres à chacune de ces activités. En parallèle, l'augmentation de la complexité accroît également la difficulté pour correctement percevoir, visualiser et analyser ces données.

Les environnement immersifs sont souvent proposées pour adresser la quantité et la complexité croissante des phénomènes modélisés, en particulier durant l'activité de visualisation. En effet, la réalité virtuelle offre une perception stéréoscopique de haute qualité utile à une meilleur compréhension de données moléculaires intrinsèquement tridimensionnelles, et permet d'afficher une quantité d'informations importante grâce aux grandes surfaces d'affichage et à la multisensorialité. Cependant, si il existe une littérature fournie sur la navigation dans les scènes virtuelles réalistes et écologiques, la navigation dans des données scientifiques abstraites est très peu étudiée. En particulier, la compréhension de phénomènes 3D complexes est fortement conditionné par la capacité du sujet à se repérer dans un phénomène 3D complexe. Une des contributions de ce travail est de proposer des paradigmes navigation adaptés au données abstraites. 

Par ailleurs, le contexte interactif des environnements immersif impose des paradigmes d'interactions directes avec les objets d'intérêt. L'usage de la réalité virtuelle se cloisonne donc à l'activité de d'exploration et de visualisation. Les autres activités supposant un contexte de travail en ligne de commande ou par l'intermédiaire de scripts adressant des outils spécifiques de simulation ou d'analyse...

L'objectif de ce travail de doctorat est donc proposer des réponses à ce besoin provenant conjointement des environnements immersif et de contextes plus classique de travail. Il s'agit de concevoir de nouvelles méthodologies susceptibles d'intégrer ces différents activités dans un contexte interactif unique, d'une part afin de minimiser les taches non productives entre activités.

Les contributions apportées dans cette thèse se base sur la représentation de la nature même des données moléculaires, des donnés analysées, des analyses, et de la nature des taches que les experts doivent accomplir. Ces deux axes de réflexion nous ont permis de mettre en place des paradigmes de navigation adaptés aux données complexes et non-orientées. De même, l'intégration des activités dans un même contexte virtuel de travail s'est fait grâce à la mise en place d'une sémantique autour des données manipulées, reliant ainsi les concepts biologiques aux concepts analytiques et permettant ainsi une communication bilatérale entre les activités.

\textbf{Mots-clefs} : Environnements immersifs, Visual Analytics, Navigation, Molecular Visualisation

%\normalsize

\begin{otherlanguage}{english}

%  \vspace{1cm}
%
%  \begin{center} \rule{\textwidth/3}{1pt} \end{center}
%  \vspace{1cm}
  
\subsection*{Abstract}
  
\footnotesize

\textbf{Keywords: Environnements immersifs, Visual Analytics, Navigation, Molecular Visualisation} 

\end{otherlanguage} 