%\skiptoevenpage 
%\addcontentsline{toc}{chapter}{Résumé / Abstract}
\backmatter 
\chapter*{}% si on enlève le \chapter , le lien dans le pdf revoit vers la biblio 
\pagestyle{empty}

\setlength{\headheight}{0.pt}
% \thispagestyle{empty} 
%\pdfbookmark[0]{Résumé}{resume}   
%~ % pour que la page soit la bonne dans la table des matière 
        
\includegraphics[height=1.cm]{./figures/UPMC_these.pdf}\hfill
\includegraphics[height=1.cm]{./figures/limsilogo_vectoriel.pdf}\hfill


\begin{center}
 \textbf{Mikael Trellet} \\
 \textbf{EXPLORATION ET ANALYSE INTERACTIVE DE DONNÉES MOLÉCULAIRES EN CONDITION IMMERSIVE}
\end{center}
    
%\fontsize{10}{12}
%\selectfont

\footnotesize
\subsection*{Résumé}

%\tiny
%(minuscule)

%\scriptsize
%(très petit)

%\footnotesize
%(assez petit)

%\small
%(petit)

%\normalsize
% normal

%\large
%(grand)

%\Large
%(plus grand)

%\LARGE
%(très grand)

%\huge
%(énorme)



\footnotesize
En biologie structurale, l'étude théorique de structures moléculaires comporte quatre activités principales organisées selon la séquence suivante : la collecte de résultats expérimentaux ou théoriques, la visualisation des structures, la simulation moléculaire, l'analyse et l'interprétation des résultats. Cette séquence permet à l'expert d'élaborer de nouvelles hypothèses, à vérifier de manière expérimentale, produisant de nouvelles données comme point de départ d'une nouvelle séquence.

L'explosion de la quantité de données à manipuler dans cette boucle pose désormais deux problèmes. Premièrement, les ressources et le temps relatifs aux tâches de transfert et de conversion de données entre chacune de ces quatre activités augmente considérablement. Deuxièmement, la complexité des données moléculaires générées par les nouvelles méthodologies expérimentales accroît fortement la difficulté pour correctement percevoir, visualiser et analyser ces données.

Les environnement immersifs sont souvent proposés pour adresser la quantité et la complexité croissante des phénomènes modélisés, en particulier durant l'activité de visualisation. En effet, la réalité virtuelle offre une perception stéréoscopique de haute qualité utile à une meilleur compréhension de données moléculaires intrinsèquement tridimensionnelles. Elle permet également d'afficher une quantité d'informations importante grâce aux grandes surfaces d'affichage et à la multisensorialité. 

Cependant, deux facteurs majeurs freinent l'usage de la réalité virtuelle dans le domaine de la biologie structurale. D'une part, même si il existe une littérature fournie sur la navigation dans les scènes virtuelles réalistes et écologiques, la navigation dans des données scientifiques abstraites est très peu étudiée. La compréhension de phénomènes 3D complexes est pourtant particulièrement conditionnée par la capacité du sujet à se repérer dans un phénomène 3D complexe. D'autre part, le contexte interactif des environnements immersif impose des paradigmes d'interaction directe avec les objets d'intérêt. L'usage de la réalité virtuelle se cloisonne donc à l'activité d'exploration et de visualisation, car les autres activités supposent un contexte de travail en ligne de commande ou par l'intermédiaire de scripts propres aux outils spécifiques de simulation et d'analyse.





% dans des contextes de travail interactifs et applicatifs indépendants,


Face à l'explosion de la quantité de données manipulées par l'utilisateur, et nouvelles problématiques précédemment décrites qui en découlent, l'objectif de ce travail de doctorat est donc, d'une part de proposer des paradigmes navigation 3D adaptés aux données moléculaires, et d'intégrer les 4 principales activités dans un contexte interactif homogène unique. 

Pour minimiser afin de minimiser les tâches non productives de gestion des données entre activité. 
Préalable à la réalisation de ces activités en interactions directes.

Les contributions répondant à ces objectifs s'appuient sur la construction d'une représentation informatique formalisant la nature des données moléculaires, les résultats d'analyse, et la nature des tâches que les experts doivent accomplir. Cette formalisation permet de concevoir des paradigmes de navigation adaptés aux contenus et à la tâche pour faciliter la compréhension de de contenus 3D complexes.  De même, l'intégration des activités contexte d'interactif plus homogène d'une sémantique autour des données manipulées, reliant ainsi les concepts biologiques aux concepts analytiques et permettant ainsi une communication bilatérale entre les activités.

\textbf{Mots-clefs} : Environnements immersifs, Visual Analytics, Navigation, Molecular Visualisation

%\normalsize

\begin{otherlanguage}{english}

%  \vspace{1cm}
%
%  \begin{center} \rule{\textwidth/3}{1pt} \end{center}
%  \vspace{1cm}
  
\subsection*{Abstract}
  
\footnotesize

\textbf{Keywords: Environnements immersifs, Visual Analytics, Navigation, Molecular Visualisation} 

\end{otherlanguage} 