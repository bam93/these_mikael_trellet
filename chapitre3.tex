\chapter[Exploration interactive de données moléculaire en immersion]{Naviguer et visualiser de façon naturelle et immersive}
\label{Sec:CantorDigitalis}
\minitoc
\cleardoublepage

Etape explicite de la boucle de biologie structurale (voir Figure \ref{Fig:schema_seq_bio_struct}), la visualisation de modèles 3d de protéines ou de complexes moléculaires permet à l'expert d'extraire de nombreuses informations de façon intuitive et rapide.
Nous avons cité plusieurs techniques de visualisation moléculaire dans la section \ref{visu_molecular} permettant de mettre en avant des informations structurelles ou physico-chimiques d'un simple coup d'oeil. Bien que ces informations soient déjà relativement complètes dans des environnements standards, elles peuvent être significativement améliorées par l'ajout de la profondeur dans la perception visuelle des données moléculaires. Il est possible de rajouter cette 3e dimension en RV grâce aux environnements immersifs et ainsi permettre une exploration plus naturelle des données.
Cependant, nous avons mis en exergue le frein que constitue la conscience spatiale altérée de l'utilisateur lors de l'exploration de complexes moléculaire dans des conditions immersives. Au-delà de la limitation spatiale que cela impose à l'utilisateur, cette conscience spatiale dégradée entraîne un malaise réduisant significativement la qualité de l'expérience de l'utilisateur. 

Pour répondre à ces problèmes issus directement de la RV, de nombreuses études ont été effectuées. Ces études sont cependant exclusivement réservées pour l'exploration de mondes virtuels écologiques, les études de navigation dans les données abstraites et scientifiques sont quant à elles beaucoup plus rares. 

Nous faisons dans ce chapitre un rapide retour sur l'exploration de données moléculaires pour ensuite comparer les méthodes de navigation dans des scènes écologiques avec les méthodes de navigation dans les scènes de données abstraites. Finalement nous présenterons nos développements et apport pour répondre au besoin de paradigmes de navigation adaptés aux données moléculaires.

\section{Navigation dans des données écologiques VS navigation dans données scientifiques abstraites}

La conception de mondes virtuels réalistes et à la taille toujours plus importante a rapidement levé la problématique de la navigation. La dissociation entre les distances pouvant être parcourues dans un monde virtuel et les contraintes dimensionnelles des dispositifs immersifs a rapidement obligé les experts en RV de mettre au point des méthodes de navigation adaptées. Ces nouvelles méthodes ne répondent pas seulement au besoin de changement d'échelle entre l'espace réel d'interaction et l'espace virtuel de navigation, elles doivent également prendre en compte la contre-productivité d'une navigation libre menant souvent à une perte de repères spatiaux. C'est par exemple le cas d'exploration immersive de structures bornées et courbées par des couloirs ou des parois que, sans retour haptique performant, l'utilisateur ne pourra que difficilement éviter. Cette perte de repères spatiaux n'est pas le seul fait de paradigmes de navigation offrant trop de liberté, elle sera également accentuée par l'abstraction des données observées. Alors que la navigation au sein d'une ville ou d'une pièce peut permettre, par l'hétérogénéité des objets/bâtiments/personnages s'y trouvant, de garder une conscience de sa position et de son orientation suffisante, la navigation dans une scène possédant une majorité d'informations abstraites et/ou non orientées diminuera cette capacité à savoir à chaque instant sa position par rapport au contenu et au monde virtuel. Les degrés de liberté de l'utilisateur pour naviguer dans sa scène virtuelle sont également un facteur pouvant compromettre sa bonne conscience spatiale. Alors que les scènes réalistes vont souvent induire une navigation avec 2 degrés de liberté parallèlement au sol virtuel, les scènes possédant des données abstraites peuvent aisément permettre une navigation en 3 dimensions dans l'ensemble du volume composant la scène virtuelle. Cela influe également l'orientation de l'utilisateur qui gardera souvent un point de vue parallèle à l'horizontale de la scène dans le cas de scènes réalistes, point de vue qui n'aura pas les mêmes contraintes lors de la navigation avec 3 degrés de liberté.

La réduction ou l'absence de repères spatiaux n'a pas seulement une conséquence sur le fait qu'un utilisateur puisse se sentir perdu au milieu d'une scène virtuelle. Elles peut également déclencher ou favoriser l'apparition d'un malaise, communément appelé \textit{cybersickness}. 

\subsection{Malaise virtuel ou "cybersickness"}

Le \textit{cybersickness} peut s'apparenter au mal des transports, transposé aux mondes virtuels et pouvant se caractériser par plusieurs effets néfastes pour l'utilisateur. En plus de simples sensations d'inconfort, on retrouve comme symptômes de la fatigue excessive, des vertiges, des maux de tête ou des nausées, tous très négatifs pour l'expérience de l'utilisateur \cite{kolasinski1995simulator,laviola_jr_discussion_2000}. Ce malaise empêche donc clairement l'efficacité de l'utilisateur pour effectuer ses tâches expertes et induit également un phénomène de "méfiance" vis-à-vis du dispositif immersif pouvant entraîner une volonté de ne pas reproduire l'expérience. Ce phénomène fut donc étudié de près afin d'en identifier les causes et d'en trouver des solutions. Les causes principales ressorties des expériences de RV menées dans le but d'induire ce phénomène passent presque exclusivement par la dissociation des canaux perceptifs du corps humain. Le découplage des informations fournies par canal visuel avec celles du système vestibulaire est particulièrement problématique. La différence de la nature des informations provenant des différents systèmes sensoriels par rapport à l'expérience usuelle de l'utilisateur a donc une chance importante d'induire ce malaise chez certaines personnes\cite{reason1975motion}. 
Typiquement, une scène virtuelle impliquant un déplacement non contrôlé de l'utilisateur et dont les paramètres de vitesse, d'accélération et de rotations ne sont pas finement paramétrés, entraînera dans beaucoup de cas un malaise de l'utilisateur. Ces situations de déplacements incontrôlés sont également responsables du mal des transports. L'absence d'implication d'un usager sur son moyen de transport, quand celui-ci possède une trajectoire et une vitesse variant de façon aléatoire, peut aussi induire un malaise. Ce phénomène de retrouve de façon négligeable lors du visionnage d'un film ou d'un contenu vidéo impliquant les mêmes déplacements mais sur un support 2d. L'immersion joue donc un rôle très important dans ce phénomène et la fidélité de restitution d'une scène réaliste impliquera souvent une probabilité de malaise plus important. Il n'est cependant pas envisageable de réduire cette immersion afin de réduire le "cybersickness", il faut donc s'intéresser à d'autres solutions. Une autre cause de malaise virtuel peut se trouver dans le taux de rafraîchissement des images affichées, plus lent que la vitesse d'analyse du cerveau créant ainsi un différentiel faisant apparaître des défauts dans l'espace d'affichage. Cette cause est souvent technique et s'explique par les ressources importantes demandées par les dispositifs immersifs et les contenus virtuels. Un taux de rafraîchissement supérieur à 40 images par seconde pour chacun des yeux (en cas de stéréoscopie active) n'est pas toujours atteignable. Parmi les pistes de solutions, nous avons vu qu'une incohérence de niveau de sollicitation du système vestibulaire de l'utilisateur par rapport à ce qu'il voit peut entraîner ce malaise. Augmenter l'implication corporelle de l'utilisateur pourrait donc constituer une réduction significative des risques d'apparition du "cybersickness".

\subsection{Méthodes de navigation dans des scènes virtuelles réalistes}

L'utilisation des moyens de navigation standard en conditions non-immersives (ordinateur de bureau, consoles de jeu, smartphones, etc.), par l'utilisation d'interacteurs fixes comme des manettes, des souris ou d'autres types de dispositifs adaptés à une utilisation à une ou deux mains dans l'espace de travail alloué au dispositif immersif ont motivé plusieurs développements, principalement lors de l'utilisation de casques virtuels, ces derniers s'utilisant bien souvent de façon fixe avec un utilisateur assis. Cependant, la majorité des développements les plus conséquents en RV sont passés par la technologie du \textit{tracking} optique afin de permettre une certaine liberté de mouvement à l'utilisateur. En plus de profiter des espaces d'évolution larges qu'offrent les systèmes CAVE ou de murs d'écrans, les solution de \textit{tracking} permettent une implication plus importante du système vestibulaire de l'utilisateur et ainsi réduisent la décorrélation entre le déplacement virtuel et le déplacement ou mouvement réel. Cette réduction entraîne en parallèle une réduction aussi importante du \textit{cybersickness} ou malaise virtuel que nous abordons plus en détail dans la section \ref{cybersickness}.

\section{Observation de complexes moléculaires de grande taille}



\section{Nouveaux paradigmes pour la RV}


