\chapter{L'ensemble musical \textit{Chorus Digitalis}}
\label{Sec:ChorusDigitalis}
%\minitoc
%\cleardoublepage
% Commentaire : la commande \texorpdfstring permet de d�clarer un titre de
% chapitre (ou section, sous-section) alternatif en texte seul, si besoin, qui
% est utilis� par hyperref pour fabriquer un menu dans les fichiers compil�s

\section{Description g�n�rale}
\lipsum[1-2]

\section{Difficult�s rencontr�es avec l'ensemble}
\lipsum[1-2]

\subsection*{Inconv�nients de notre instrument}
\lipsum[1-2]

\section{Le r�pertoire des concerts}
\lipsum[1-2]

\subsection{Liste des concerts}
\lipsum[1-2]

\subsection{Chorale classique europ�enne}
\label{sec:chorus-choral}
\noindent \textit{Voir fichier audio~/ vid�o~\ref{fav:choral-JAS}}\\
\lipsum[1-2]

\subsection{Chant vocal d'Inde du nord}
\label{sec:chorus-raga}
\noindent \textit{Voir fichiers audios~/ vid�os~\ref{fav:raga}}\\
\lipsum[1-2]

\subsection{Chorale contemporaine}
\label{sec:chorus-contempo}
\noindent \textit{Voir fichier audio~/ vid�o~\ref{fav:valse}}\\
\lipsum[1-2]


