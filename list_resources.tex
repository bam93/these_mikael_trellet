\chapterstar{Liste des fichiers audio-visuels}
% Pour des renseignements sur \chapterstar : voir le fichier macros.tex

\noindent Les fichiers audios et vidéos listés ci-desous sont disponibles à l'adresse suivante :\\ \url{http://groupeaa.limsi.fr}\\
ou sur simple demande au groupe \textit{Audio et Acoustique} du LIMSI-CNRS.

\begin{enumerate}


\section*{Chapitre~\ref{Sec:CantorDigitalis} : Cantor Digitalis}


\item    Analyse synthèse d'un /a/ (partie~\ref{sec:cantor-example-voyelle})
\label{fav:natsyn}
		\begin{itemize}
		    \item Voix naturelle
       		\item Voix synthétique (sans perturbation automatique)
        \end{itemize}
\item    Atténuation des amplitudes des filtres formantiques avec F0 (partie~\ref{sec:cantor-ai-attenuation})
\label{fav:ai-attenuation}
		\begin{itemize}
        \item Glissando sans atténuation
        \item Glissando avec atténuation
        \end{itemize}
\item    Dépendance de F1 avec l'effort vocal (partie~\ref{sec:cantor-f1f0})
\label{fav:f1-VE-dependance}
		\begin{itemize}
        \item Crescendo sans dépendance
        \item Crescendo avec dépendnace
         \end{itemize}
\item    Dépendance des fréquences {F1,F2,F3,F4,F5} des filtres formantiques avec F0 (partie~\ref{Sec:Fi_F0})
\label{fav:fi-f0-dependance}
		\begin{itemize}
        \item Glissando sans dépendance
        \item Glissando avec dépendance
        \end{itemize}
\item    Types de voix 1 (partie~\ref{Sec:tailleConduit})
\label{fav:types-voix-1}
		\begin{itemize}
        \item Soprano
       \item Alto
        \item Ténor
        \item Basse
        \item Bébé
        \item Enfant
        \end{itemize}
\item   Types de voix 2 (partie~\ref{sec:cantor-voixMonstres})
\label{fav:types-voix-2}
		\begin{itemize}        
        \item Soprano bruité
        \item Alto bruité
        \item Fauve
        \item Monstre
         \end{itemize}
\item Types de voix 3 (partie~\ref{sec:cantor-chantDiphonique})
\label{fav:types-voix-3}
		\begin{itemize}        
        \item Chant diphonique
		\end{itemize}
\newpage
\section*{Chapitre~\ref{Sec:digitartic} : Digitartic}

 \item   Quelques syllabes (partie~\ref{sec:dynPhaseArti})
 \label{fav:syllabes}
 	\begin{itemize}
 		\item apa,ata,aka,ava,aza,a3a,awa,aya,aja,ama,ana/ 
 	\end{itemize}
 \item  Degré et vitesse d'articulation (partie~\ref{Sec:hypoarticulation})
 \label{fav:degreesArti}
 		\begin{itemize}        
        \item Différents degrés d'articulation avec /aja/ 
        \item Vitesse de contrôle articulatoire avec /awa/
        \end{itemize}
 \item   Démonstration générale 
 \label{fav:demo-expressivite}
 		\begin{itemize}
 		\item Expressivité de l'articulation (partie~\ref{sec:ConclusionDigitartic})
 		\end{itemize}

\section*{Annexe~\ref{Sec:ChorusDigitalis} : L'ensemble Chorus Digitalis}

\item Classique européen
\label{fav:choral-JAS}
	\begin{itemize}
		\item  Vidéo du choral «~Alta Trinita Beata~» de Bach @PS3 workshop, Vancouver (partie~\ref{sec:chorus-choral})	
	\end{itemize}
\item   Raga d'Inde du Nord (partie~\ref{sec:chorus-raga})
\label{fav:raga}
 	\begin{itemize}
 		\item Journée Sciences et Musique 2012, Rennes (Audio)
 		\item Journées Art Science 2012, Printemps de la Culture, Orsay (Vidéo)
 	\end{itemize}
\item  Contemporain
\label{fav:valse} 
	\begin{itemize}
		\item Vidéo de la polyphonie contemporaine «Valse~» (Bruno Lecossois) @Journées Art Science 2012, Printemps de la Culture, Orsay (partie~\ref{sec:chorus-contempo})
	\end{itemize}
\item  Autre
  	\begin{itemize}
  		\item Vidéo intégrale du concert du Printemps de la Culture 2012, moins les morceaux «~Ocean~» et «~North Star~» n'ayant pas l'autorisation des auteurs.
  	\end{itemize}
\end{enumerate}



