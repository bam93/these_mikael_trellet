\chapter[Exploration interactive de données moléculaire en immersion]{Naviguer et visualiser de façon naturelle et immersive}
\minitoc
\cleardoublepage

\section{Introduction}
Ils ont raison : \cite{Klatt80} \cite{Holmes83} \cite{VertegaalUngKie96} \cite{Henrich01} \cite{Cadoz94}.

L'étude des nombreuses voies métaboliques de la cellule permettant son fonctionnement repose en grande partie sur l'analyse des molécules qui y sont impliquées. Elles sont en effet les actrices des cascades de réaction qui caractérisent ces voies métaboliques. La connaissance de la  

\section{L'appareil vocal, émetteur sonore}
\begin{eqnarray}
\left\{ \begin{array}{ll}
	F_1(V\!E=0.8)=&F1 \\
	F_1(V\!E=0.6)=&0.925F1\\
	F_1(V\!E=1.0)=&1.075F1\\
\end{array} \right.
\end{eqnarray}

\begin{eqnarray}
F_1(V\!E)=
\left\{
	\begin{array}{lll}
		&(0.7+0.375 \cdot V\!E) \cdot F1& \mbox{ si $V\!E\geq0.6$}\\
		&0.925 \cdot F1& \mbox{ si $V\!E<0.6$}\\
	\end{array}
\right.
\end{eqnarray}

\begin{equation} 
\begin{split}
b_{TL} & = 1 - \nu + \sqrt{\nu^2-1}\\
\nu    & = 1-\frac{1}{\eta} \\
\eta   & = \frac{10^{TL/10}-1}{\cos(2\pi\frac{3000}{F_e})-1}
\end{split} 
\end{equation} 

\begin{eqnarray} 
\vert H( e^{2i \pi fc /F_e}) \vert = \frac{\vert H(1) \vert}{ \sqrt[]{2}}  \label{Eqa:1}
\end{eqnarray}


\noindent En élevant au carré (\ref{Eqa:1}), on obtient :
 
\begin{eqnarray} 
	\frac{b_{TL}^2}{2 \cdot (1-b_{TL})(1-\cos(2i \pi f_c / F_e)) + b_{TL}^2} &=& \frac{1}{2}
\end{eqnarray}




\subsection{Description anatomique de l'appareil vocal}


\begin{figure}
  \centering
  \subfloat
  [{\it Sans dépendance}]
  {\includegraphics[width=.45\linewidth]{ch2/fig/Fi-F0-dpdce-sans_soprano_u_200-1500Hz_dig13b3.pdf}}
  \label{Fig:Fi-F0-dpdce_sans}
  \hspace{0.3cm}
  \subfloat
	[{\it Avec dépendances}]  
  {\includegraphics[width=.45\linewidth]{ch2/fig/Fi-F0-dpdce-avec_soprano_u_200-1500Hz_dig13b3.pdf}}
  \label{Fig:Fi-F0-dpdce_avec}
    \caption{{\it Spectrogrammes de la voyelle /a/ de synthèse, où $F_0$ augmente avec le temps, (a) sans ou (b) avec les dépendances entre les fréquences centrales des formants et de $F_0$. \textit{Voir fichiers audios~/ vidéos~\ref{fav:fi-f0-dependance}}\\
}}
  \label{Fig:Fi-F0-dpdce}
\end{figure}

\subsubsection{a) La soufflerie}

\begin{table}[!h]
	\centering
	\begin{tabular}{|c|c|c|} 
		\hline
		& \centering \textbf{Tessiture naturelle moyenne} & \centering \textbf{Tessiture
dans le synthétiseur} \tabularnewline
		\hline
		\bf Basse & Mi2-Mi4 & Sol$\sharp$1-Sol4\\
		\hline
		\bf Ténor & Do3-Si4 & Sol$\sharp$1-Sol4\\
		\hline
		\bf Alto & Fa3-Mi5 & Sol$\sharp$2-Sol5\\
		\hline
		\bf Soprano & Si3-Do6 & Sol$\sharp$3-Sol6\\
		\hline
	\end{tabular}
	\caption{\textit{Tessiture des chanteurs naturels et synthétiques (La3=440 Hz). Voir fichiers audios~/ vidéos~\ref{fav:types-voix-1}}}
	\label{Tab:tessChant}
\end{table}

\subsubsection{b) Le larynx}

\subsubsection{c) Le conduit vocal}



\subsection{Description source-filtre de l'appareil vocal}

\begin{figure}
  \centering
  {\includegraphics[width=1.\linewidth]{ch1/fig/ModeleSourceFiltre.pdf}}
    \caption{{\it Représentation schématique du fonctionnement du modèle source-filtre}}
  \label{Fig:ModeleSourceFiltre}
  \hspace{0.3cm}
\end{figure}


\subsection{Description acoustique de l'appareil vocal}

\subsubsection*{a) L'effort vocal}
\lipsum[1-1]
\subsubsection*{b) La dimension tendue / relâchée}
\lipsum[1-1]
\subsubsection*{c) Les apériodicités}
\lipsum[1-1]
\subsubsection*{d) Le phonétogramme}
\lipsum[1-1]
\subsubsection*{e) La notion de registre ou mécanisme laryngé }


\subsubsection*{f) Les formants}
\lipsum[1-1]

\section{La voix comme objet de synthèse pour le jeu musical}
\lipsum[1-1]

\subsection{Méthodes de synthèse vocale}
\lipsum[1-1]

\subsubsection{a) La synthèse par formants}
\label{sec:synthFormants}
\lipsum[1-1]

\subsubsection{b) La synthèse par concaténation}
\lipsum[1-1]

\subsubsection{c) La synthèse par HMM}
\lipsum[1-1]

\subsubsection{d) La synthèse par modèle physique}
\lipsum[1-1]

\subsection{Quelques instruments de synthèse vocale pour le jeu musical}
\label{sec:qqIntrus}
\lipsum[1-1]

\subsubsection{a) Machines mécaniques}
\label{Sec:macMec}
\lipsum[1-1]

\subsubsection{b) Le VODER}
\lipsum[1-1]

\subsubsection{c) Le MUSSE}
\lipsum[1-1]

\subsubsection{d) Le SPASM}
\lipsum[1-1]

\subsubsection{e) Glove-Talk II}
\lipsum[1-1]

\subsubsection{f) Contrôle gestuel du programme CHANT}
\lipsum[1-1]

\subsubsection{g) Le Voicer}
\lipsum[1-1]

\subsubsection{h) Le HandSketch}
\lipsum[1-1]
